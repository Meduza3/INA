\documentclass{article}
\usepackage[utf8]{inputenc}
\usepackage[T1]{fontenc}
\let\lll\undefined % Usunięcie istniejącej definicji
\usepackage{amssymb}
\usepackage{amsmath}
\usepackage{graphicx}
\usepackage{algorithm}
\usepackage{algorithmic}
\usepackage[polish]{babel}

\title{aod-lab-lista4}
\author{Marcin Zubrzycki}
\date{January 2025}

\begin{document}

\maketitle


\section*{Algorytm Edmondsa-Karpa}
\subsection*{Opis struktury grafu}
Rozważany graf jest hiperkostką $H_k, k\in \{1,...,16\}$, czyli grafem którego zbiorem wierzchołków jest zbiór liczb $\{0,...,2^{k}-1\}$. Wierzchołki połączone są ze sobą tylko jeśli zapis binarny ich indeksów rózni się na dokładnie jednej pozycji krawędzią skierowaną z wierzchołka o mniejszej liczbie jedynek do tego z większą. Pojemności przyjmowane przez krawędzie są losowane jednostajnie z przedziału $\{1,...,2^l\}$, gdzie $l$ równe jest największej z czterech wartości: ilość zer lub ilość jedynek z dowolnego spośród dwóch zamieszanych wierzchołków.
\subsection*{Opis Algorytmu}
Algorytm Edmondsa-Karpa służy do znajdowania maksymalnego przepływu w sieci przepływowej. Implementuje on metodę Forda-Fulkersona, w której wybór ścieżki powiększającej jest dokonywany za pomocą metody Breadth-First-Search. 
Po znalezieniu najkrótszej ścieżki puszczamy przepływ tą ściężką i szukamy kolejnej ścieżki w następnej iteracji, aż zapełnią się wszystkie drogi od źródła. Wtedy przepływ puszczony w sieci jest maksymalną ilością przepływu od źródła do ujścia.
\begin{algorithm}[H]
  \caption{Edmonds-Karp}
  \label{alg:edmonds-karp}
  \begin{algorithmic}[1]
  \REQUIRE Graf przepływowy $G=(V,E)$, przepustowości $c(u,v)$, źródło $s$, ujście $t$
  \ENSURE Maksymalny przepływ od $s$ do $t$
  \STATE Inicjalizuj przepływ $f(u,v) \leftarrow 0$ dla każdej krawędzi $(u,v) \in E$.
  \WHILE{istnieje w grafie rezydualnym ścieżka $P$ z $s$ do $t$ znaleziona za pomocą BFS}
    \STATE Wyznacz minimalną rezydualną przepustowość $\text{cf}_{\min}$ na ścieżce $P$, 
           tj. $\text{cf}_{\min} = \min_{(u,v) \in P} \bigl(c(u,v) - f(u,v)\bigr)$.
    \FORALL{krawędzi $(u,v)$ należących do ścieżki $P$}
      \STATE $f(u,v) \leftarrow f(u,v) + \text{cf}_{\min}$ 
      \STATE $f(v,u) \leftarrow f(v,u) - \text{cf}_{\min}$ \quad (aktualizacja przepływu rewersyjnego)
    \ENDFOR
  \ENDWHILE
  \STATE \textbf{return} $f$ \quad (wielkość maksymalnego przepływu to $\sum_{v\in V} f(s,v)$)
  \end{algorithmic}
  \end{algorithm}
\subsection*{Metodologia Eksperymentu}
Testy zostały przeprowadzone na hiperkostkach $H_k$, gdzie $k\in\{1,...,16\}$. Dla każdego rozmiaru kostki eksperyment był uruchamiany 10krotnie. Wykres prezentuje średni maksymalny przepływ i średni czas działania programu. 
\end{document}
