\documentclass{article}
\usepackage[T1]{fontenc}
\usepackage[polish]{babel}
\usepackage{amsmath}
\usepackage{algpseudocode}
\usepackage{graphicx}
\usepackage{bm}

\begin{document}
\section{Zadanie 1}
\section{Zadanie 2}
\section{Zadanie 3}
\section{Zadanie 4}
Dla dowolnego wielomianu $p(n)$ i dowolnej stałej $c$ istnieje liczba całkowita $n_0$, taka że dla każdego $n \ge n_0$ zachodzi $2^{cn} \ge p(n)$.
Dowolny wielomian będzie rósł z prędkością wielomianową, tj dla wielomianu $p(n)$ stopnia $k$ istnieje stała $M > 0$ taka, że dla wystarczającu dużych $n$ zachodzi $p(n) \leq Mn^k$.
Wzrost wykładniczy dominuje wielomianowy, stąd $\lim\limits_{n\rightarrow\infty}\frac{2^{cn}}{n^k}=\infty$. Czyli istnieje takie $n_0$ że dla każdego $n$ większego zachodzi $\frac{2^{cn}}{n^k} \geq M$, czyli wtedy $2^{cn} \geq Mn^k \geq p(n)$
\begin{itemize}
  \item $p(n) = n^2$, $c=1$: $n_0=1$
  \item
\end{itemize}
\section{Zadanie 5}
\section{Zadanie 6}
\end{document}
